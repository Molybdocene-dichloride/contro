\documentclass[hidelinks,12pt]{article}

\usepackage{geometry}

\usepackage{fontspec}
\usepackage{unicode-math}
\usepackage{polyglossia}

\usepackage{ifthen}

\usepackage{amsmath}

\usepackage{mhchem}
\usepackage{chemfig}

\usepackage{graphicx}
\usepackage{pdfpages}

\usepackage{setspace}

\usepackage[explicit]{titlesec}
\usepackage{titletoc}
\usepackage{tocloft}

\usepackage{multirow}
\usepackage{listings}
\usepackage{array}
\usepackage{varwidth}

\usepackage[hidelinks]{hyperref}

\usepackage{titloid}

\renewcommand{\baselinestretch}{1}

\geometry{a4paper, portrait, margin=30mm, rmargin=15mm, bmargin=20mm, tmargin=20mm}
%\setlength{\textheight}{2mm}

\usetikzlibrary{decorations.markings}

\setmainlanguage{russian}

\setmainfont{Times New Roman}
\setmathfont{Times New Roman}

\setstretch{1.5}

\renewcommand\cftsecfont{\normalfont}
\renewcommand\cftsecpagefont{\normalfont}
\renewcommand{\cftsecleader}{\cftdotfill{\cftsecdotsep}}
\renewcommand\cftsecdotsep{\cftdot}
\renewcommand\cftsubsecdotsep{\cftdot}

\newfontfamily\sectionfont{Times New Roman}

\titleformat{\section}{\fontsize{12}{12}\bfseries\sectionfont}{\hspace{12.5mm} \thesection \quad \MakeUppercase{#1}}{.5em}{}
\titleformat{\subsection}{\fontsize{12}{12}\sectionfont}{\hspace{12.5mm} \thesubsection \quad #1}{.5em}{}

\renewcommand{\cfttoctitlefont}{\hspace*{\fill}\Large}
\renewcommand{\cftaftertoctitle}{\hspace*{\fill}}

\makeatletter
\addto\captionsrussian@modern{\renewcommand{\contentsname}{\fontsize{12pt}{12pt}\selectfont ОГЛАВЛЕНИЕ}}
\makeatother

\hypersetup{
  hidelinks,
  frenchlinks=true,
  colorlinks=false,
  pdfborder={0 0 0},
  citecolor=black,
  filecolor=black,
  linkcolor=black,
  urlcolor=black
}
  
\begin{document}
  \includepdf[pages=-]{titul.pdf}

  \newpage

  \begin{center} \uppercase{Лист для замечаний} \end{center}
  
  \newpage
  
  \setlength{\parskip}{0pt}
  \tableofcontents

  \newpage
  \newcounter{prevformula} \setcounter{prevformula}{0}
  \TitleStyle{Задание}
  
  \mStyle{Напишите уравнения реакций, указав условия их протекания, побочные продукты при их наличии и назвав органические вещества, а также приведите их механизмы:}

  \mStyle{1.1 Монохлорирование изобутана}
  
  \mStyle{1.2 Бромирование циклопентена}
  
  \mStyle{1.3 Гидробромирование гексадиена-1,3}
  
  \mStyle{1.4 Мононитрование метилбензола}
  
  \newpage
  
  \newcolumntype{P}[1]{>{\centering\arraybackslash}p{#1}}
  \newcounter{title} \setcounter{title}{1} \newcounter{chem} \setcounter{chem}{1}
  
  \def\baselinestretch{1.5}\fontsize{12pt}{12pt}\selectfont
  \setcharge{extra sep =5pt}
  \reaction{Монохлорирование изобутана}{
  \mStyle{При монохлорировании 2-метилпропана (изобутана) при нагревании в качестве основного продукта образуется 2-метил-1-хлорпропан, а в качестве побочного продукта -\\ 2\nobreakdash-метил\nobreakdash-2\nobreakdash-хлорпропан:}
    
    \formul{\ce{\chemfig{-[:30](-[:90])-[:-30]} ->[Cl_2][\parbox{12px}{\fontsize{8pt}{8pt}\selectfont $t°$ -$HCl$}] \chemfig{-[:30](-[:90])-[:-30](-[:30]Cl)} + \chemfig{-[:30](-[:90])(-[:30]Cl)-[:-30]}}}
    \mStyle{При монохлорировании 2-метилпропана (изобутана) при облучении ультрафиолетовым излучением в качестве основного продукта образуется 2-метил-2-хлорпропан, а в качестве побочного продукта - 2-метил-1-хлорпропан:}
    \formul{\ce{\chemfig{-[:30](-[:90])-[:-30]} ->[Cl_2][\parbox{12px}{\fontsize{8pt}{8pt}\selectfont $hv$ -$HCl$}] \chemfig{-[:30](-[:90])(-[:30]Cl)-[:-30]} + \chemfig{-[:30](-[:90])-[:-30](-[:30]Cl)}}}
    
  }{\mStyle{Реакция протекает по механизму радикального замещения ($S_R$) как цепной радикальный процесс и включает следующие стадии:}
    \mStyle{– инициирование цепи:}
    
    \formul{\ce{Cl_2 -> Cl^{.} + Cl^{.}}}
    
    \mStyle{– развитие цепи:}
    
    \formul{\ce{\chemfig{-[:30](-[:90])-[:-30]} ->[Cl^{.}][-HCl] \chemfig{-[:30]\charge{-90=\.}{}(-[:90])-[:-30]}}}
    
    \formul{\ce{\chemfig{-[:30](-[:90])-[:-30]} ->[Cl^{.}][-HCl] \chemfig{-[:30](-[:90])-[:-30]\charge{-90=\.}{}}}}

    \formul{\ce{\chemfig{-[:30]\charge{-90=\.}{}(-[:90])-[:-30]} ->[Cl_2][-Cl^{.}] \chemfig{-[:30](-[:90])(-[:30]Cl)-[:-30]}}}

    \formul{\ce{\chemfig{-[:30](-[:90])-[:-30]\charge{-90=\.}{}} ->[Cl_2][-Cl^{.}] \chemfig{-[:30](-[:90])-[:-30](-[:30]Cl)}}}
    
    \mStyle{– обрыв цепи:}

    \formul{\ce{\chemfig{-[:30]\charge{-90=\.}{}(-[:90])-[:-30]} ->[Cl^{.}] \chemfig{-[:30](-[:90])(-[:30]Cl)-[:-30]}}}

    \formul{\ce{\chemfig{-[:30](-[:90])-[:-30]\charge{-90=\.}{}} ->[Cl^{.}] \chemfig{-[:30](-[:90])-[:-30](-[:30]Cl)}}}

    \formul{\ce{Cl^{.} + Cl^{.} -> Cl_2}}
  }

\newpage

\reaction{Бромирование циклопентена}{
  \mStyle{При бромировании циклопентена образуется 1,2-дибромциклопентан}
  \formul{\ce{\chemfig{*5(-=---)} ->[Br_2][] \chemfig{*5(-(-[:-55]Br)-(-[:5]Br)---)}}}
}{\mStyle{Реакция протекает по механизму электрофильного присоединения ($A_E$) и включает следующие стадии:}
  \mStyle{– образование π-комплекса:}
  
  \formul{\ce{\chemfig{*5(-=---)} <-->[Br_2][] \chemfig{?-[:90]-[:20]-[:-55]=[:-125,.5](-[:-30,0.7,,,->]Br-[:-30,0.7,,,,]Br)=[:-125,.5]?}}}
  
  \mStyle{– образование σ-комплекса (карбкатиона, медленная, скорость-лимитирующая стадия):}
  
  \formul{\ce{\chemfig{?-[:90]-[:20]-[:-55]=[:-125,.5](-[:-30,0.7,,,->]Br-[:-30,0.7,,,]Br)=[:-125,.5]?} ->[][\text{$-Br^⊖$}] \chemfig{*5(-\charge{-90=$⊕$}{}-(-[:5]Br)---)} <--> \chemfig{*5(-(-[:-55]Br)-\charge{0=$⊕$}{}---)} <--> \chemfig{*5(-*3(-Br\charge{80:5pt=$⊕$}{}-)----)}}}
  
  \mStyle{– образование продукта:}
  
  \formul{\ce{\chemfig{*5(-\charge{-90=$⊕$}{}-(-[:5]Br)---)} ->[\text{$Br^⊖$}] \chemfig{*5(-(-[:-55]Br)-(-[:5]Br)---)}}}
}

\newpage

\pgfdeclaredecoration{ddbond}{initial}{
  \state{initial}[width=4pt]{
    \pgfpathlineto{\pgfpoint{4pt}{0pt}}
    \pgfpathmoveto{\pgfpoint{2pt}{2pt}}
    \pgfpathlineto{\pgfpoint{4pt}{2pt}}
    \pgfpathmoveto{\pgfpoint{4pt}{0pt}}
  }
  \state{final}{
    \pgfpathlineto{\pgfpointdecoratedpathlast}
  }
}
\tikzset{lddbond/.style={decorate, decoration=ddbond}}
\tikzset{rddbond/.style={decorate, decoration={ddbond, mirror}}}

\reaction{Гидробромирование гексадиена-1,3}{
  \mStyle{При гидробромировании гексадиена-1,3 в качестве основных продуктов образуются 3-бромгексен-1, 1\nobreakdash-бромгексен\nobreakdash-2, 2\nobreakdash-бромгексен\nobreakdash-3 и  4-бромгексен\nobreakdash-2, а в качестве побочных продуктов - 4\nobreakdash-бромгексен\nobreakdash-1 и 1\nobreakdash-бромгексен\nobreakdash-3}
  \formul{\ce{\chemfig{=[:30]-[:-30]=[:30]-[:-30]-[:30]} ->[HBr][] \chemfig{=[:30]-[:-30](-[:-90]Br)-[:30]-[:-30]-[:30]} + \\ + \chemfig{(-[:150]Br)-[:30]=[:-30]-[:30]-[:-30]-[:30]} + \chemfig{-[:30](-[:90]Br)-[:-30]=[:30]-[:-30]-[:30]} + \\ + \chemfig{-[:30]=[:-30]-[:30](-[:90]Br)-[:-30]-[:30]} + \chemfig{=[:30]-[:-30]-[:30](-[:90]Br)-[:-30]-[:30]} + \\ + \chemfig{(-[:150]Br)-[:30]-[:-30]=[:30]-[:-30]-[:30]}}}
}{\mStyle{Реакция протекает по механизму электрофильного присоединения ($A_E$) и включает следующие стадии:}
 \mStyle{– образование π-комплекса:}
  
  \formul{\ce{\chemfig{=[:30]-[:-30]=[:30]-[:-30]-[:30]} <-->[HBr] \chemfig{=[:30]-[:-30]=[:30,.5](-[:-30,0.7,,,->]H-[:-30,0.7,,,]Br)=[:30,.5]-[:-30]-[:30]}}}

  \formul{\ce{\chemfig{=[:30]-[:-30]=[:30]-[:-30]-[:30]} <-->[HBr] \chemfig{=[:30,.5](-[:-30,0.7,,,->]H-[:-30,0.7,,,]Br)=[:30,.5]-[:-30]=[:30]-[:-30]-[:30]}}}
  
  \mStyle{– образование σ-комплексов (карбкатионов аллильного типа), стабилизированных резонансом (медленная, скорость-лимитирующая стадия):}
  
  \formul{\ce{\chemfig{=[:30]-[:-30]=[:30,.5](-[:-30,0.7,,,->]H-[:-30,0.7,,,]Br)=[:30,.5]-[:-30]-[:30]} ->[][\text{$-Br^⊖$}] \chemfig{=[:30]-[:-30]\charge{-90=$⊕$}{}-[:30]-[:-30]-[:30]} <-> \\\\ <-> \chemfig{\charge{-90=$⊕$}{}-[:30]=[:-30]-[:30]-[:-30]-[:30]} \text{≡} \chemfig{\text{\charge{-90:4pt=$\frac{1}{2}⊕$}{}}-[:30,,,,,rddbond]-[:-30,,,,,rddbond]\text{\charge{-90:4pt=$\frac{1}{2}⊕$}{}}-[:30]-[:-30]-[:30]} \text{≡} \\\\ \text{≡} \chemfig{-[:30,,,,,rddbond]\text{\charge{-90:4pt=$⊕$}{}}-[:-30,,,,,rddbond]-[:30]-[:-30]-[:30]} }}
%\\\\
  \formul{\ce{\chemfig{=[:30,.5](-[:-30,0.7,,,->]H-[:-30,0.7,,,]Br)=[:30,.5]-[:-30]=[:30]-[:-30]-[:30]} ->[][\text{$-Br^⊖$}] \chemfig{-[:30]\charge{90=$⊕$}{}-[:-30]=[:30]-[:-30]-[:30]} <-> \\\\\\ <-> \chemfig{-[:30]=[:-30]-[:30]\charge{90=$⊕$}{}-[:-30]-[:30]} \text{≡} \chemfig{-[:30]\text{\charge{90:4pt=$\frac{1}{2}⊕$}{}}-[:-30,,,,,lddbond]-[:30,,,,,lddbond]\text{\charge{90:4pt=$\frac{1}{2}⊕$}{}}-[:-30]-[:30]} \text{≡} \\\\ \text{≡} \chemfig{-[:30]-[:-30,,,,,lddbond]\charge{90:4pt=$⊕$}{}-[:30,,,,,lddbond]-[:-30]-[:30]} }}

  \mStyle{– образование побочных σ-комплексов (карбкатионов), нестабилизированных резонансом (медленная, скорость-лимитирующая стадия):}

  \formul{\ce{\chemfig{=[:30]-[:-30]=[:30,.5](-[:-30,0.7,,,->]H-[:-30,0.7,,,]Br)=[:30,.5]-[:-30]-[:30]} ->[][\text{$-Br^⊖$}] \chemfig{=[:30]-[:-30]-[:30]\charge{90=$⊕$}{}-[:-30]-[:30]}}}

  \formul{\ce{\chemfig{=[:30,.5](-[:-30,0.7,,,->]H-[:-30,0.7,,,]Br)=[:30,.5]-[:-30]=[:30]-[:-30]-[:30]} ->[][\text{$-Br^⊖$}] \chemfig{\charge{-90=$⊕$}{}-[:30]-[:-30]=[:30]-[:-30]-[:30]}}}
  
  \mStyle{– образование продуктов 1,2- и 1,4-присоединения:}
  
  \formul{\ce{\chemfig{=[:30]-[:-30]\charge{-90=$⊕$}{}-[:30]-[:-30]-[:30]} ->[\text{$Br^⊖$}] \chemfig{=[:30]-[:-30](-[:-90]Br)-[:30]-[:-30]-[:30]}}}

 \formul{\ce{\chemfig{\charge{-90=$⊕$}{}-[:30]=[:-30]-[:30]-[:-30]-[:30]} ->[\text{$Br^⊖$}] \chemfig{(-[:150]Br)-[:30]=[:-30]-[:30]-[:-30]-[:30]}}}

  \formul{\ce{\chemfig{-[:30]\charge{90=$⊕$}{}-[:-30]=[:30]-[:-30]-[:30]} ->[\text{$Br^⊖$}] \chemfig{-[:30](-[:90]Br)-[:-30]=[:30]-[:-30]-[:30]}}}

\formul{\ce{\chemfig{-[:30]=[:-30]-[:30]\charge{90=$⊕$}{}-[:-30]-[:30]} ->[\text{$Br^⊖$}] \chemfig{-[:30]=[:-30]-[:30](-[:90]Br)-[:-30]-[:30]}}}

  \mStyle{– образование побочных продуктов 1,2-присоединения:}

  \formul{\ce{\chemfig{=[:30]-[:-30]-[:30]\charge{90=$⊕$}{}-[:-30]-[:30]} ->[\text{$Br^⊖$}] \chemfig{=[:30]-[:-30]-[:30](-[:90]Br)-[:-30]-[:30]}}}

  \formul{\ce{\chemfig{\charge{-90=$⊕$}{}-[:30]-[:-30]=[:30]-[:-30]-[:30]} ->[\text{$Br^⊖$}] \chemfig{(-[:150]Br)-[:30]-[:-30]=[:30]-[:-30]-[:30]}}}
}


\newpage

\reaction{Мононитрование метилбензола}{
  \mStyle{При мононитровании метилбензола (толуола) в сернокислом растворе в качестве основного продукта образуется орто-нитротолуол (1-метил-2\nobreakdash-нитробензол), а в качестве побочных продуктов - пара\nobreakdash-нитротолуол (1\nobreakdash-метил-4\nobreakdash-нитробензол) и мета\nobreakdash-нитротолуол\\ (1\nobreakdash-метил-3\nobreakdash-нитробензол)}
  \formul{\ce{\chemfig{**6(-(-[:-90])-----)} ->[$\fontsize{13pt}{13pt}\selectfont\mathrm{HNO_3}$][\parbox{16px}{\fontsize{8pt}{8pt}\selectfont$H_2SO_4$ -$H_2O$}] \chemfig{**6(-(-[:-90])-(-[:-30]NO_2)----)} + \chemfig{**6(-(-[:-90])---(-[:90]NO_2)--)} + \chemfig{**6(-(-[:-90])--(-[:30]NO_2)---)}}}
}{\mStyle{Реакция протекает по механизму электрофильного замещения ($S_E$) и включает следующие стадии:}
  \mStyle{– активация электрофила:}
  
  \formul{\ce{2HNO_3 <--> H_2NO_3^$⊕$ + NO_3^$⊖$ <-->[][-H_2O] NO_2^$⊕$ + NO_3^$⊖$}}

  \formul{\ce{HNO_3 <-->[H_2SO_4][\text{$-HSO_4^⊖$}] H_2NO_3^$⊕$ <-->[][-H_2O] NO_2^$⊕$}}
  
  \mStyle{– образование π-комплекса:}
  
  \formul{\ce{\chemfig{**6(-(-[:-90])-----)} <-->[\text{$NO_2^⊕$}] \chemfig{**6(-(-[:-90])-----)}\chemmove{
    \node[at=(cyclecenter1)](){}
    node [at=(cyclecenter1),shift=(0:1.9cm)](end){\printatom{\text{$NO_2^⊕$}}};
    \draw[->,shorten <=.5cm](cyclecenter1)--(end);
}}}
  \mStyle{– образование σ-комплексов, стабилизированных резонансом:}
  
  \formul{\ce{\chemfig{**6(-(-[:-90])-----)}\chemmove{%
    \node[at=(cyclecenter1)](){}
    node [at=(cyclecenter1),shift=(0:1.9cm)](end){\printatom{\text{$NO_2^⊕$}}};
    \draw[->,shorten <=.5cm](cyclecenter1)--(end);
  }\phantom{aawww} ->\phantom{aa} \chemfig{*6(=(-[:-90])-=-(-[:90]NO_2)-\charge{-180=$⊕$}{}-)} <-> \chemfig{*6(-\charge{-50=$⊕$}{}(-[:-90])-=-(-[:90]NO_2)-=)} <-> \chemfig{*6(-(-[:-90])=-\charge{0=$⊕$}{}-(-[:90]NO_2)-=)}\phantom{a}\text{≡} \\ \text{≡}
  \chemfig{**[60,-240]6(-(-[:-90])---(-[:90]NO_2)--)} \chemmove{%
    \node[at=(cyclecenter1)](){$⊕$};
  }}}

\formul{\ce{\chemfig{**6(-(-[:-90])-----)}\chemmove{%
    \node[at=(cyclecenter1)](){}
    node [at=(cyclecenter1),shift=(0:1.9cm)](end){\printatom{\text{$NO_2^⊕$}}};
    \draw[->,shorten <=.5cm](cyclecenter1)--(end);
  }\phantom{aawww} ->\phantom{aa} \chemfig{*6(-\charge{-50=$⊕$}{}(-[:-90])-(-[:-30]NO_2)=-=-)} <-> \chemfig{*6(-(-[:-90])=(-[:-30]NO_2)-\charge{0=$⊕$}{}(-[:-90])-=-)} <->\\<-> \chemfig{*6(-(-[:-90])=(-[:-30]NO_2)-=-\charge{-180=$⊕$}{}-)} \text{≡}
  \chemfig{**[-10, 305]6(-(-[:-90])-(-[:-30]NO_2)----)} \chemmove{%
    \node[at=(cyclecenter1)](){$⊕$};
  }}}

\formul{\ce{\chemfig{**6(-(-[:-90])-----)}\chemmove{%
    \node[at=(cyclecenter1)](){}
    node [at=(cyclecenter1),shift=(0:1.9cm)](end){\printatom{\text{$NO_2^⊕$}}};
    \draw[->,shorten <=.5cm](cyclecenter1)--(end);
  }\phantom{aawww} -> \chemfig{*6(=(-[:-90])-=(-[:30]NO_2)-\charge{90=$⊕$}{}--)} <-> \phantom{aa} \chemfig{*6(=(-[:-90])-\charge{0=$⊕$}{}-(-[:30]NO_2)=--)} <->\\<->  \phantom{aa} \chemfig{*6(\charge{180=$⊕$}{}-(-[:-90])=-(-[:30]NO_2)=--)} \text{≡}
  \chemfig{**[55, 370]6(-(-[:-90])--(-[:30]NO_2)---)} \chemmove{%
    \node[at=(cyclecenter1)](){$⊕$};
  }}}
  \mStyle{– образование продуктов за счет выброса протона, сопровождающегося восстановлением ароматичности:}
  
  \formul{\ce{\chemfig{*6(=(-[:-90])-=-(-[:90]NO_2)-\charge{-180=$⊕$}{}-)} ->[$\mathrm{H_2O}$][$\mathrm{-H_3O^⊕}$] \chemfig{**6(-(-[:-90])---(-[:90]NO_2)--)}}}

  \formul{\ce{\chemfig{*6(-\charge{-50=$⊕$}{}(-[:-90])-(-[:-30]NO_2)=-=-)} ->[$\mathrm{H_2O}$][$\mathrm{-H_3O^⊕}$] \chemfig{**6(-(-[:-90])-(-[:-30]NO_2)----)}}}

  \formul{\ce{\chemfig{*6(=(-[:-90])-=(-[:30]NO_2)-\charge{90=$⊕$}{}--)} ->[$\mathrm{H_2O}$][\text{$\mathrm{-H_3O^⊕}$}] \chemfig{**6(-(-[:-90])--(-[:30]NO_2)---)}}}
  
}

\end{document}
